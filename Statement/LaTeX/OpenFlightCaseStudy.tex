\documentclass{report}

\usepackage[utf8]{inputenc}
\usepackage[T1]{fontenc}
\usepackage{listings}
\lstloadlanguages{R}
\usepackage[french]{babel}
\usepackage{graphicx}
\usepackage{fontawesome}
\usepackage[dvipsnames]{xcolor}
\usepackage[framemethod=TikZ]{mdframed}
\usepackage{amsmath}
\usepackage{amsfonts}
\usepackage{amssymb}

% Creation of a command to generate the title page
\newcommand*{\titlePage}{
	\begingroup 
		\hbox{
			\hspace*{0.05\textwidth}
			\rule{1.5pt}{\textheight}
			\hspace*{0.05\textwidth}
			\parbox[b]{0.75\textwidth}{
				{\noindent\LARGE\bfseries Étude de cas:} \\[\baselineskip]
				{\noindent\huge\bfseries Analyse de marché du} \\
				{\noindent\huge\bfseries transport aérien canadien} \\
				{\noindent\huge\bfseries avec R} \\[2\baselineskip]
				% Title
				{\large\textit{Atelier d'introduction à R}}\\[4\baselineskip]
				% Author name
				{\Large\textsc{Beauchemin, David}} \\[\baselineskip]
				{\Large\textsc{Cabral Cruz, Samuel}} \\[\baselineskip]
				{\Large\textsc{Goulet, Vincent}} \\[4\baselineskip]
				% Association
				{\large {Dans le cadre du colloque R à Québec}} \\[\baselineskip]
				{\large {25 mai 2017}} 
				\vspace{0.425\textheight}}}\endgroup}

% Default Addition of Pictures
\graphicspath{{./fig/}}
\newcommand{\AddPicture}[4]{
	\begin{figure}[h]
		\begin{center}
			\includegraphics[width=#1\textwidth, height=#2\textheight,keepaspectratio]{#3}
			\caption{#4}
		\end{center}
	\end{figure}}
	
% Creation of environment to add additional informations
\mdfsetup{
	linecolor=NavyBlue,
	linewidth=2pt,
	backgroundcolor=NavyBlue!10,
	roundcorner=10pt}	
\newenvironment{moreInfo}[1]
	{\begin{mdframed}
	\textcolor{NavyBlue}{\huge \raisebox{-3.5pt}{\faInfo} 
	\hspace{0.5cm} \large\bfseries #1}\\[5pt]
	\normalsize
	\makebox[0.1\textwidth][l]{}	
	\begin{minipage}{10cm}}
	{	\end{minipage}
	\end{mdframed}}

% Lstlisting configurations
\definecolor{dkgreen}{rgb}{0,0.6,0}
\definecolor{gray}{rgb}{0.5,0.5,0.5}
\definecolor{mauve}{rgb}{0.58,0,0.82}
\lstset{
	language=R,                     
	basicstyle=\footnotesize,       
	numbers=left,                   
	numberstyle=\tiny\color{gray},  
	stepnumber=1,                   
	numbersep=5pt,                  
	backgroundcolor=\color{white},  
	showspaces=false,               
	showstringspaces=false,         
	showtabs=false,                 
	frame=single,                   
	rulecolor=\color{black},        
	tabsize=2,                      
	captionpos=b,                   
	breaklines=true,                
	breakatwhitespace=false,        
	title=\lstname,                 
	keywordstyle=\color{blue},      
	commentstyle=\color{dkgreen},   
	stringstyle=\ttfamily\color{mauve},
	%identifierstyle=\color{magenta},
	escapeinside={\%*}{*)},   
	morekeywords={*,...}} 

\begin{document}
\begin{titlepage}
	\clearpage\thispagestyle{empty}
	\titlePage
\end{titlepage}
\tableofcontents
\chapter*{Préface}
\addcontentsline{toc}{chapter}{Préface}

\chapter*{Introduction}
\addcontentsline{toc}{chapter}{Introduction}

\chapter*{Étude de cas}
\addcontentsline{toc}{chapter}{Étude de cas}
\setcounter{chapter}{1}

\section{Extraction, traitement, visualisation et analyse des données}
	%LaTeX file for Q1 section
HELLO

\section{Création de fonctions utilitaires}

\section{Communication des résultats}

\section{Analyse de la compétition}

\section{Ajustement de distribution statistiques sur données empiriques}

\section{Simulation et analyse de rentabilité}

\chapter*{Conclusion}
\addcontentsline{toc}{chapter}{Conclusion}

% Examples
\begin{moreInfo}{setwd()}
	La fonction setwd permet de redéfinir la valeur du working directory
\end{moreInfo}
\begin{lstlisting}
#### Setting working directory properly ####
setwd('..')
(path <- getwd())
set.seed(31459)
#### Question 1 - Extraction, traitement, visualisation et analyse des donnees ####
# 1.1 - Extraire les bases de donnees airports.dat et routes.dat
airports <- read.csv("https://raw.githubusercontent.com/jpatokal/openflights/master/data/airports.dat", header = FALSE, stringsAsFactors = TRUE, na.strings=c('\\N',''))
routes <- read.csv("https://raw.githubusercontent.com/jpatokal/openflights/master/data/routes.dat", header = FALSE, stringsAsFactors = TRUE, na.strings=c('\\N',''))
airlines <- read.csv("https://raw.githubusercontent.com/jpatokal/openflights/master/data/airlines.dat", header = FALSE,  stringsAsFactors = TRUE, na.strings=c('\\N',''))
\end{lstlisting}

\end{document}
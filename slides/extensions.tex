\section{Extensions du système de base}

\begin{frame}
  \frametitle{Bibliothèques et paquetages}

  \begin{itemize}
  \item Fonctions, données et documentation réunies dans des
    \alert{paquetages} (\emph{packages})
  \item Paquetages réunis dans une \alert{bibliothèque}
    (\emph{library})
  \item Bibliothèque de base de R contient une trentaine de paquetages
    dont certains sont chargés par défaut au démarrage
  \item Fonction pour charger un paquetage: \texttt{library}
  \end{itemize}
\end{frame}

\begin{frame}
  \frametitle{Bibliothèque personnelle}

  \begin{alertblock}{Astuce}
    Créez-vous une bibliothèque personnelle tel qu'expliqué à
    l'annexe~D de %
    \href{https://github.com/vigou3/introduction-programmation-r/releases/download/edition-5a/introduction-programmation-r.pdf}{Goulet (2016)}.
  \end{alertblock}
\end{frame}

\begin{frame}
  \frametitle{Ajout d'extensions}

  \begin{itemize}
  \item \emph{Comprehensive R Archive Network} (CRAN) dépôt central de
    paquetages R
    \begin{quote}
      \url{https://cran.r-project.org}
    \end{quote}
  \item Installation automatisée depuis R avec
    \texttt{install.packages}
  \item Après installation, charger le paquetage avec \texttt{library}
  \end{itemize}

  \pause
  \gotoR{extensions.R}
\end{frame}

%%% Local Variables:
%%% mode: latex
%%% TeX-engine: xetex
%%% TeX-master: "raquebec-atelier-introduction-r"
%%% End:

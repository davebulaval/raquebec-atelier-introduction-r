Cette annexe est destinée à fournir des indications sur comment installer R sur votre poste de travail ainsi que les outils qui faciliteront vos débuts avec ce langage de programmation. 

\section{Installation de R}
Le première étape consiste à l'installation de R. Pour ce faire, rendez-vous sur \emph{Comprehensize R Archive Network} (\href{https://cran.r-project.org/}{CRAN}). Cliquer ensuite sur le lien correspondant à votre système d'exploitation. \footnote{Il est fortement recommandé de télécharger une distribution binaire précompilée. Cette annexe ne traitera que de cette manière de procéder.} Il suffira ensuite de suivre les instructions fournies qui varieront selon votre système d'exploitation. Ces dernières sont résumées ci-dessous.

\subsection{\emph{Windows}}
Pour les utilisateurs \emph{Windows}, il suffira de cliquer sur \textcolor{blue}{\emph{install R for the first time}} et ensuite \textcolor{blue}{\emph{Download R x.y.z for Windows }}. Une fois l'\emph{executable} (\texttt{.exe}) téléchargé, il vous suffira de l'exécuter et de suivre les instructions d'installation. \\

Nous vous recommandons de modifier le chemin d'installation par défaut \newline (\verb|C:\Program Files\R\R-x.y.z|) pour \verb|C:\R\R-x.y.z|. Les raisons de cette modification sont:
\begin{itemize}
	\item Éviter de devoir toujours exécuter R avec les privilèges administrateur afin d'installer des paquetages dans la librairie principale.
	\item Éviter les problèmes qui peuvent provenir de la présence d'espace dans les chemins d'installation.
\end{itemize}

\subsection{\emph{(MAC) OS X}}
Si vous opérer un \emph{MAC}, vous n'aurez qu'à télécharger et exécuter le \emph{package} (\verb|R-x.y.z.pkg|) de la version la plus récente. 

\subsection{\emph{Linux or Unix-Like}}
R est présentement disponible pour les distributions suivantes:\begin{itemize}
	\item Debian
	\item RedHat
	\item SUSE
	\item Ubuntu
\end{itemize}
Choississez votre distribution et suivez les instructions indiquées. Si vous êtes sur une distribution qui n'est pas supportée, il sera nécessaire de compiler R à partir du code source. Cette procédure est décrite à la section 2.5.1 de la \emph{Frequently Asked Question list} (\href{https://cran.r-project.org/doc/FAQ/R-FAQ.html#How-can-R-be-installed-_0028Unix_002dlike_0029}{R FAQ}).

\section{Installation de R Studio}
Une fois le programme R installer, bien qu'il sera possible d'utiliser R via son interface standard, il sera préférable d'utiliser un \emph{Integrated Development Environment} (IDE) pour créer des scripts qui pourront facilement être conservés et/ou partagés. Si vous êtes peu habitués à ce genre d'outils, il sera conseillé d'utiliser R Studio. R Studio est un logiciel libre et gratuit. Rendez-vous directement sur le site de R Studio pour procéder à son installation. \\
\url{https://www.rstudio.com/products/rstudio/download2/} \\

Prendre note que ceux qui seraient déjà familiers avec un autre éditeur, tel que Emacs, peuvent très bien conserver leurs habitdes et éviter l'installation d'un IDE spécifique pour chaque langage de programmation.

\section{Installation de paquetages}
L'installation de paquetages R supplémentaires à ceux offerts par défaut pourra se faire de plusieurs manières différentes. Les deux premières méthodes sont très manuelles et ne sont présentées qu'à titre informatif. Il sera beaucoup plus efficient d'utiliser la troisième méthode. \\

Lorsque nous sommes dans l'interface standard de R, il suffira de cliquer sur l'onglet \emph{Packages} et d'ensuite sélectionner \emph{Install package(s)...}. Une fenêtre vous demandera ensuite de choisir un \emph{mirror}. Nous conseillons de prendre \emph{0-Cloud [https]}. Une autre fenêtre s'ouvrira enfin et vous pourrez choisir le paquetage désiré en trouvant son nom dans la liste. \\

De manière similaire, sous R Studio, vous pourrez aller à l'onglet \emph{Tools} et sélectionner \emph{Install Packages...}. Une fenêtre apparaîtera et vous pourrez procéder à une recherche du paquetage désiré. \\

Finalement, vous pourrez toujours installer des paquetages directement via la console avec la commande R \texttt{install.packages("<nom paquetage>")}. \\

Une fois les paquetages installés, n'oubliez pas d'utiliser la commande \texttt{library(<nom paquetage>)} avant toute utilisation de fonctions rendues accessibles sous ces derniers. \footnote{L'installation d'un paquetage donné ne devra se faire qu'une seule fois lors de sa première utilisation.}

\section{Mise à jour de R et de R Studio}
La mise à jour de R est maintenant beaucoup plus simple grâce au paquetage \texttt{installr}. Une fois la librairie appelée par l'environnement, la fonction \texttt{updateR} sera accessible. Par son appel, une vérification sera fait à savoir si une version plus récente de R est disponible. Le cas échéant, une fenêtre de navigation apparaîtra afin de vous guider dans le processus de mise à jour. Autrement, puisque la fonction \texttt{updateR} n'est qu'une fonction mère réalisant les appels nécessaire dans un ordre précis, vous pourrez préciser toutes vos réponses comme montré par le \autoref{src:updateR}. Vous pourrez aussi remarqué qu'à la mise à jour de R, on vous demandera si vous désirez mettre à jour vos paquetages. \cite{updateR} \\

\begin{lstlisting}[caption = Mise à jour de R avec la fonction \texttt{updateR},label=src:updateR]
library(intallr)
# install, move, update.package and quit R
updateR(F, T, T, F, T, F, T)
\end{lstlisting}

\vspace{\baselineskip}
Pour la mise à jour de R Studio, vous n'aurez qu'à aller sous l'onglet \emph{Help} pour ensuite sélectionner \emph{Check for Updates}. Si une nouvelle version de R Studio est disponible, on vous demandera si vous êtes intéressé à la télécharger et en acceptant, votre navigateur web s'ouvrira automatiquement sur la page de téléchargement de R Studio où vous aurez accès au nouveau contenu. 

\section{Gestionnaire de versions}
Tout developpment devrait être fait à l'aide d'un gestionnaire de versions. Ce sujet est très large et hors d'application du document. Nous nous contentons de donner des liens vers des références intéressantes à ce sujet.

compte privés
version étudiante
https://github.com/
https://git-scm.com/download
https://www.youtube.com/playlist?list=PL-osiE80TeTuRUfjRe54Eea17-YfnOOAx
tutoriels youtube très intéressants
lien vers formation que je suis en train de faire.
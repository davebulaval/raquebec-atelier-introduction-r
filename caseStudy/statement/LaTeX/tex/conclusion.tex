Au terme de cette étude de cas, nous avons su intégrer différentes notions relatives à la programmation en R. Nous avons abordé des sujets aussi variés qu'actuels allant de l'importation des données jusqu'à l'analyse par simulation. \\

Cette formation n'a jamais eu la prétention de pouvoir vous apprendre toutes les particularités du langage R ni même faire de vous des programmeurs parfaitement fonctionnels au terme de sa lecture. Par contre, nous croyons avoir bel et bien accompli l'objectif principal qui était d'étaler au grand jour les capacités de R tout en vous offrant un coffre d'outils qui facilitera grandement vos débuts avec ce langage. Il n'y a pas de secret pour apprendre à programmer, mais il existe certainement des moyens plus efficaces que d'autres. Selon nous, une connaissance adéquate de ce que l'on peut ou pas réaliser consiste en un excellent point de départ. Par après, à un moment ou un autre, vous serez confronté à un problème qui vous semblera parfaitement adapté à l'utilisation de R. Vous chercherez ensuite à accumuler les ressources et connaissances nécessaires à sa résolution. \footnote{Évidemmemnt, rien ne vous empêche de vous créer des problèmes fictifs comme nous l'avons fait avec cette étude.} Ce n'est qu'en mettant en pratique vos connaissances que la maîtrise du langage sera atteignable. La route sera souvent cahoteux, mais le résultat donc bien satisfaisant. \\

Par ailleurs, dans une ère aussi axée sur le développement informatique et l'automatisation des tâches, il est de plus en plus important d'avoir de connaissances tangibles dans le domaine de la programmation. L'apprentissage du langage R est sans aucun doute une très bonne idée en raison de sa facilité d'accès, de la taille de sa communauté et de sa simplicité. Comme nous pouvons le voir à la \autoref{fig:redmonk}, R est toujours un langage d'actualité très prisé qui en vaut le détour en se plaçant au $12^{ième}$ rang selon le classement \emph{RedMonk} \cite{codingGame}. \\

\addPicture{1}{0.5}{redmonkProgLanguagePopularity}{Classement \emph{RedMonk} des différents langages de programmation}{redmonk}

Lorsque nous regardons la croissance du nombre de paquetages disponibles sur R à travers les années à la \autoref{fig:RpckAvailableGrowth}, nous constatons qu'elle se rapproche d'une croissance exponentielle pour désormais mettre à la disposition des utilisateurs plus de 10 000 paquetages. Il sera pertinent de faire quelques recherches sur CRAN avant de vous lancer dans le développement d'une quelconque fonctionnalité avec autant de choix. \\

\addPicture{1}{0.35}{RpckAvailableGrowth}{Nombre de paquetage disponible sur CRAN dans le temps \cite{scrappingCRAN}}{RpckAvailableGrowth}

En raison du caractère libre du langage R, ce dernier a toujours été et restera en perpétuel développement. C'est la raison principale pourquoi nous parlons toujours de ce langage à l'heure actuelle, tandis que plusieurs autres sont tombés dans les oubliettes. Par contre, un des principes fondamentaux du développement libre implique la coopération de ses utilisateurs. Si nous profitons de ce que la communauté nous apporte, nous devrions aussi être en mesure de contribuer à la communauté lorsque nous pensons avoir réalisé une tâche qui pourra intéresser et être récupérée par d'autres utilisateurs. \\

En ce qui nous concerne, sans l'accès aux données d'\emph{OpenFlights}, la totalité de cette étude n'aurait pu être réalisée. C'est pour cette raison que nous trouvions important d'apporter notre part à ce projet au meilleur de nos connaissances. Il nous est venu à l'idée de partager la méthodologie que nous avons empruntée pour corriger l'information des fuseaux horaires. Nous tentons par ce transfert de connaissances à la communauté d'éviter à un prochain utilisateur de devoir se questionner sur la validité et la complétion de cette information comme nous l'avons fait. Vous pourrez consulter l'\autoref{ann:contribOpenFlights} pour plus d'informations au sujet de notre contribution. \\

Le développement ou la modification du code source d'un projet n'est pas la seule manière de contribuer. Plusieurs projets libres se fondent principalement sur la récolte et l'aggrégation des connaissances de la communauté. C'est notamment les fondations du projet \emph{OpenStreetMap} qui a été bâti par une communauté de cartographes bénévoles. Avec ce projet, quiconque peut désormais s'improviser cartographe et apporter sa part en allant directement sur leur outil de cartographie en ligne à l'adresse suivante: \url{http://www.openstreetmap.org/#map=0/-145/56}. Le plus merveilleux dans tout ceci, c'est que vous pourrez acquérir de nouvelles compétences tout en travaillant pour le bien commun de la société. \\

Il est désormais beaucoup plus facile de comprendre pourquoi le succès d'un projet libre résidera dans sa capacité à rendre la contribution accessible à tous. C'est pour cette raison que plusieurs ressources seront mises à la disposition des utilisateurs pour expliquer comment procéder pour faire leur part. À titre d'exemple, nous avons personnellement suivi les instructions de la vidéo suivante afin d'apprivoiser l'outil: \url{https://www.youtube.com/watch?v=Ir-3K0pjwOI}. Après seulement quelques minutes, nous avons été en mesure de produire la cartographie représentée à la \autoref{fig:osmChildhood}. Et maintenant, pourquoi pas commencer par la cartographie de votre quartier d'enfance?! \\

\addPicture{1}{0.5}{osmChildhood}{Exemple de contribution alternative}{osmChildhood}

En guise de conclusion, je tiens à remercier David Beauchemin et Vincent Goulet pour leur support tout au long de l'écriture de ce document et sans qui je ne serais certainement pas parvenu à composer le tout dans une si petite période de temps. Je tiens aussi à souligner l'importante contribution de Keven Migneault pour avoir volontairement lu d'une couverture à l'autre le document et avoir améliorer de manière significative la qualité de son édition. À vous, chers, acolytes, en espérant retravailler avec vous dans un avenir rapproché!
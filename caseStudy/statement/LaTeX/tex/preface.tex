Dans le cadre du colloque "R à Québec" qui s'est tenu le 25 et 26 mai 2017 sur le campus de l'Université Laval, une séance d'introduction au langage de programmation R fut offerte aux participants. Cette séance visait principalement la maîtrise des rudiments de cet environnement de programmation tout en prenant conscience des capacités de ce langage. \cite{RQC2017} Elle sera divisée en deux parties. En ce qui concerne la première partie, les fondements du langage seront visités d'une manière théorique sous la forme d'un exposé magistral. La deuxième partie, tant qu'à elle, se concentrera davantage sur la mise en pratique des notions abordées lors de la première partie grâce à la complétion d'une analyse de marché du transport aérien canadien. Ce document correspond en fait à la documentation complète de cette deuxième partie de formation. \\

Étant donné qu'il s'agit tout de même d'une formation pour débutants, la majorité du code sera déjà fournie, mais il n'en vaut pas moins la peine de parcourir ce projet si ce n'est que pour constater la puissance et la simplicité du langage. De plus, il est souvent difficile de mettre en perspective les innombrables fonctionnalités d'un langage lorsque nous commençons à l'utiliser. Cette étude de cas nous fournit ainsi un bel exemple d'enchaînement de traitements jusqu'à son aboutissement ultime qui consiste à faire une analyse de compétitivité. \\

D'autre part, il est important de préciser que le code qui sera présenté ne correspond pas toujours à la manière la plus efficiente d'accomplir une tâche donnée. L'objectif principal étant ici la transmission de connaissances dans un dessin éducatif plutôt que d'une réelle analyse de marché. Il est aussi important de mentionner que, bien qu'il s'agisse d'une formation s'adressant à des débutants, plusieurs notions qui seront mises en valeur font plutôt état de niveau intermédiaire ou avancé, mais toujours apportées de manière simplifiée et accessible à n'importe qui n'ayant jamais travaillé avec R. \\

Nous tenons à remercier Vincent Goulet de nous avoir fait confiance dans l'élaboration de cette partie de la formation ainsi que tous les membres du comité organisationnel de l'évènement. Nous croyons sincèrement que R est un langage d'actualité qui se révèlera un atout à quiconque oeuvrant dans un domaine relié de près ou de loin aux mathématiques.
\subsection{Visualisation et analyse des données}
\label{subsec:Visualisation}

La visualisation des données est une étape cruciale dans l’interprétation de celles-ci. En effet, seule une connaissance approfondie des données vous permettra d'en percer les secrets les plus précieux qui y résident. Afin de visualiser les données directement contenues dans une relation, le langage R met à notre disposition différentes fonctions qui sont décrites ci-dessous. 

\begin{description}[style=multiline,leftmargin=2cm]
		\item[\texttt{View}] Permet d'ouvrir un \texttt{data.frame} dans l'outil de visualisation de RStudio. Ce dernier permettra aussi d'appliquer des transformations de faible complexité comme le filtre sur une variable ou le tri. \cite{Rfunction:View}
		\item[\texttt{head}] Renvoie en console un extrait des premières observations d'une relation (par défaut, 10 observations sont renvoyées). \cite{Rfunction:head}
		\item[\texttt{summary}] Compilation de statistiques pertinentes au sujet des différentes variables contenues dans une table. Pour les variables quantitatives, le minimum, le $1^{er}$ quintile, la moyenne, la médiane, le $3^{ième}$ quintile et le maximum seront calculés, tandis qu'une simple analyse de fréquence des différentes modalités sera produite dans le cas d'une variable qualitative.
		\item[\texttt{table}] Au même titre que le comportement de \texttt{summary} pour les variables qualitatives, la fonction \texttt{table} renvoie un vecteur comptabilisant le nombre d'occurrences de chaque valeur unique. \cite{Rfunction:table}
\end{description}

\begin{lstlisting}[caption = Fonctions de visualisation de données,label=src:DataVisual]
View(airportsCanada)
head(airportsCanada)
summary(airportsCanada)
nbAirportCity <- table(airportsCanada$city) 
\end{lstlisting}

\vspace{\baselineskip}
Ces fonctions ressemblent beaucoup plus à des outils pour optimiser le temps de développement qu'à des traitements que nous chercherons à laisser en production compte tenu de leur affichage très peu conviviale et pratique. De plus, plus la relation possède de variables, plus il est facile de s'y perdre. Pour contrer ces problèmes, la production de graphiques sera, la plupart du temps, une solution plus qu'intéressante. Cependant, toujours dans un objectif de cohérence avec la structure du code source du projet, nous n'aborderons pas immédiatement la création de graphiques en R. Nous nous contenterons plutôt d'introduire les méthodes de visualisation de données géospatiales pour faire le pont avec la \autoref{subsec:Traitement}.\\

Au moment de l'analyse, deux paquetages ont retenu notre attention pour la production de cartes géographiques qui faciliteront la transmission de connaissances sommaires au sujet du jeu de données. Nos critères de sélection étaient encore une fois la simplicité des requêtes, la beauté de l'extrant final et la flexibilité des instructions pour les adapter à un contexte précis. \\

Le paquetage \texttt{ggmap}, nous a permis de produire la \autoref{fig:ggmapTrafic}. Si cette dernière vous semble familière, ce n'est pas sans raison! Le paquetage \texttt{ggmap} vise en fait à combiner la visualisation de données géospatiales sur support statique disponible en ligne, tels que ceux de \emph{Google Maps}, avec la puissance du paquetage \texttt{ggplot2}. \cite{pkgR:ggmap} \\

\addPicture{1}{0.5}{ggmapTrafic}{Exemple de carte géographique produite avec \texttt{ggmap}}{ggmapTrafic} 

En jetant un coup d'oeil au \autoref{src:ggmap}, nous voyons qu'il est possible de produire des cartes très rapidement avec seulement quelques lignes de code. Malgré la facilité d'utilisation de \texttt{ggmap}, nous ressentons vite ses limitations lorsque nous espérons produire des cartes interactives similaires à celles que nous retrouvons dans la plupart des applications web et mobiles modernes. \\

\begin{lstlisting}[caption = Générer une carte du trafic aérien avec \texttt{ggmap},label=src:ggmap]
# install.packages("ggmap")
library(ggmap)
map <- get_map(location = "Canada", zoom = 3)
TraficData <- subset(airportsCanada,as.numeric(paste(combinedIndex)) > 0.05)
lon <- as.numeric(paste(TraficData$longitude))
lat <- as.numeric(paste(TraficData$latitude))
size <- as.numeric(paste(TraficData$combinedIndex))
airportsCoord <- as.data.frame(cbind(lon, lat, size))
mapPoints <- 
  ggmap(map) + 
  geom_point(data=TraficData,aes(x=lon,y=lat,size=size),alpha=0.5,shape=16)
(mapTraffic <-  
  mapPoints + 
  scale_size_continuous(range = c(0, 20),name = "Trafic Index"))
\end{lstlisting}

\vspace{\baselineskip}
Pour ce faire, le paquetage \texttt{leaflet} \cite{leaflet} viendra à notre secours avec un faible coût en complexité compte tenu de la flexibilité impressionnante rajoutée. Ce paquetage n'est rien d'autre qu'une enveloppe permettant de faire appel à la librairie \emph{JavaScript}. \cite{leafletjs} Le \autoref{src:leaflet} est à l'origine de la \autoref{fig:leafletTrafic} provenant en fait d'une carte interactive. \\

\addPicture{1}{0.4}{leafletTrafic}{Exemple de carte géographique produite avec \texttt{leaflet}}{leafletTrafic}

\begin{lstlisting}[caption = Générer une carte du trafic aérien avec \texttt{leaflet},label=src:leaflet]
# install.package("leaflet")
library(leaflet)
url <- "http://hiking.waymarkedtrails.org/en/routebrowser/1225378/gpx"
download.file(url, destfile = paste(path,"/Reference/worldRoutes.gpx",sep=""), method = "wget")
worldRoutes <- readOGR(paste(path,"/Reference/worldRoutes.gpx",sep=""), layer = "tracks")
markersData <- subset(airportsCanada,IATA %in% c('YUL','YVR','YYZ','YQB'))
markersWeb <- c("https://www.aeroportdequebec.com/fr/pages/accueil",
                "http://www.admtl.com/",
                "http://www.yvr.ca/en/passengers",
                "https://www.torontopearson.com/")

# Defining the description text to be displayed by the markers
descriptions <-paste("<b><FONT COLOR=#31B404> Airport Details</FONT></b> <br>",
                    "<b>IATA: <a href=",markersWeb,">",markersData$IATA,"</a></b><br>",
                    "<b>Name:</b>",markersData$name,"<br>",
                    "<b>Coord.</b>: (",markersData$longitude,",",markersData$latitude,") <br>",
                    "<b>Trafic Index:</b>",markersData$combinedIndex)

# Defining the icon to be add on the markers from fontawesome library
icons <- awesomeIcons(icon = 'paper-plane',
                      iconColor = 'black',
                      library = 'fa')

# Combinaison of the different components in order to create a standalone map
(mapTraffic <- leaflet(worldRoutes) %>%
    addTiles(urlTemplate = "http://{s}.basemaps.cartocdn.com/light_all/{z}/{x}/{y}.png") %>%
    addCircleMarkers(stroke = FALSE,data = TraficData, ~as.numeric(paste(longitude)), ~as.numeric(paste(latitude)),
                     color = 'black', fillColor = 'green',
                     radius = ~as.numeric(paste(combinedIndex))*30, opacity = 0.5) %>%
    addAwesomeMarkers(data = markersData, ~as.numeric(paste(longitude)), ~as.numeric(paste(latitude)), popup = descriptions,icon=icons))

# Resizing of the map
mapTraffic$width <- 874
mapTraffic$height <- 700

# Export of the map into html format
# install.packages("htmlwidgets")
library(htmlwidgets)
saveWidget(mapTraffic, paste(path,"/Reference/leafletTrafic.html",sep = ""), selfcontained = TRUE)
\end{lstlisting}

\vspace{\baselineskip}
Le fonctionnement des deux paquetages est sensiblement le même. Nous commençons par extraire une carte qui servira de support directement à partir du web. Nous passons ensuite les informations géographiques nécessaires au constructeur du paquetage. Nous ajoutons ensuite des composantes à cette instance à l'aide de méthode conçue spécifiquement à cette fin. Sans entrer davantage dans les détails, il est intéressant de mentionner les particularités que le paquetage \texttt{leaflet} offre en sus des fonctionnalités graphiques traditionnelles. \\

Tout d'abord, les \texttt{markers} peuvent être personnalisés de fond en comble. Dans l'exemple présent, nous avons mis à profit la banque de symboles et d'outils CSS (\emph{Cascading Style Sheets}) \emph{fontawesome} \cite{fontAwesome} qui est célèbre auprès des utilisateurs \LaTeX pour la diversité et la qualité de ses icônes. Un autre aspect encore plus pratique est la présentation d'informations supplémentaires lorsque l'utilisateur appuie sur le marqueur offrant ainsi une manière simple de stocker beaucoup d'information au sein du même objet sans alourdir indûment sa lisibilité. L'ajout de ces informations et le formatage se résument par le passage de commandes \emph{html} directement à l'argument \texttt{popup}. Vous savez maintenant comment nous avons procédé pour exposer le code IATA, le nom, les coordonnées géographiques ainsi que l'indice de trafic aérien sur chacun des marqueurs auxquels l'icône \texttt{fa-paper-plane} a été assigné. Le dernier point intéressant de \texttt{leaflet} est la capacité de créer des \emph{widgets html} indépendants rendant le partage de l'information encore plus simple sans nécessiter de recompiler le code source à chaque fois qu'un utilisateur aura envie de visionner l'objet. \cite{leaflet} \\

\begin{moreInfo}{Est-ce tout ce que peut accomplir \texttt{leaflet}?}
	Bien entendu, les exemples présentés dans ce document ne font l'éloge que de deux applications grossières de ces deux paquetages. Vous serez en mesure de trouver plusieurs autres exemples sur le web. Pour l'instant, voici quelques pages d'intérêt qui ont servi à créer la carte interactive: \\
	\url{https://rstudio.github.io/leaflet/} \\
	\url{https://rstudio.github.io/leaflet/markers.html} \\
	\url{https://rstudio.github.io/leaflet/popups.html} \\
	\url{http://rgeomatic.hypotheses.org/550} \\
	\url{https://www.r-bloggers.com/interactive-mapping-with-leaflet-in-r/} \\
	\url{http://stackoverflow.com/questions/38837112/how-to-specify-radius-units-in-addcirclemarkers-when-using-leaflet-in-r} \\
	\url{http://stackoverflow.com/questions/31562383/using-leaflet-library-to-output-multiple-popup-values} \\
	\url{https://gis.stackexchange.com/questions/171827/generate-html-file-with-r-using-leaflet}
\end{moreInfo}

En terminant, il est possible de valider nos résultats en comparant ceux-ci avec la densité de la population canadienne. Nous devrions être en mesure d'observer une augmentation du trafic aérien dans les zones où la densité de population est plus appréciable (Voir \autoref{fig:popDensityCanada}). 

\addPicture{1}{0.4}{popDensityCanada}{\href{http://www.huffingtonpost.ca/2014/04/17/canada-empty-maps_n_5169055.html}{Densité de la population canadienne}}{popDensityCanada}
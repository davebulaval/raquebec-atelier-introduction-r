Au terme de cette étude de cas, nous avons su intégrer différentes notions relatives à la programmation en R. Nous avons abordé des sujets aussi variés qu'actuels allant de l'importation des données jusqu'à l'analyse par simulation. \\

Cette formation n'a jamais eu la prétention de pouvoir vous apprendre toutes les particularités du langage R ni même faire de vous des programmeurs parfaitement fonctionnels au terme de sa lecture. Par contre, nous croyons avoir bel et bien accompli l'objectif principal qui était d'étaler au grand jour les capacités de R tout en vous offrant un coffre d'outils qui facilitera grandement vos débuts avec ce langage. Il n'y a pas de secret pour apprendre à programmer, mais il existe certainement des moyens plus efficaces que d'autres. Selon nous, une connaissance adéquate de ce que l'on peut ou pas réaliser consiste en un excellent point de départ. Par après, à un moment ou un autre, vous serez confronté à un problème qui vous semblera parfaitement adapté à l'utilisation de R. Vous chercherez ensuite à accumuler les ressources et connaissances nécessaires à sa résolution. \footnote{Évidemmemnt, rien ne vous empêche de vous créer des problèmes fictifs comme nous l'avons fait avec cette étude} Ce n'est qu'en mettant en pratique vos connaissances que la maîtrise du langage sera atteignable. La route sera souvent tortueuse, mais le résultat donc bien satisfaisant. \\

Par ailleurs, dans une ère aussi axée sur le développement informatique et l'automatisation des tâches, il est de plus en plus important d'avoir de connaissances tangibles dans le domaine de la programmation. L'apprentissage du langage R est sans aucun doute une très bonne idée en raison de sa facilité d'accès, de la taille de sa communauté et de sa simplicité. Comme nous pouvons le voir à la \autoref{fig:redmonk}, R est toujours un langage d'actualité très prisé qui en vaut le détour en se plaçant au $12^{ième}$ rang selon le classement \emph{RedMonk} \cite{codingGame}. \\

\addPicture{1}{0.5}{redmonkProgLanguagePopularity}{Classement \emph{RedMonk} des différents langages de programmation}{redmonk}

En raison du caractère libre du langage R, ce dernier a toujours été et restera en perpétuel développement. C'est la raison principale pourquoi nous parlons toujours de ce langage à l'heure actuelle, tandis que plusieurs autres sont tombés dans les oubliettes. Par contre, un des principes fondamentaux du développement libre implique la coopération de ses utilisateurs. Si nous profitons de ce que la communauté nous apporte, nous devrions aussi être en mesure de contribuer à la communauté lorsque nous pensons avoir réalisé une tâche qui pourra intéresser et être récupérée par d'autres utilisateurs. En ce qui nous concerne, sans l'accès aux données d'\emph{OpenFlights}, la totalité de cette étude n'aurait pas pu être réalisée. En travaillant avec ces données, nous avons eu à faire un peu de reconstitution au niveau des fuseaux horaires. C'est pour cette raison qu'une contribution de notre part sera effectuée directement via le GitHub du projet \emph{OpenFlights} pour regarnir la variable \texttt{tzFormat}. (Voir \autoref{ann:contribOpenFlights})\\

En guise de conclusion, je tiens à remercier David Beauchemin et Vincent Goulet pour leur support tout au long de l'écriture de ce document et sans qui je ne serais certainement pas parvenu à composer le tout dans une si petite période de temps. À vous chers, acolytes, en espérant retravailler avec vous dans un avenir rapproché!